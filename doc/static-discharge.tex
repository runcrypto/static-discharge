\documentclass[a4paper]{article}

\usepackage{listings}
\usepackage{hyperref}
\usepackage{xcolor}
\usepackage{listings}

% define colors used in this style
\definecolor{cornellred}{rgb}{0.7, 0.11, 0.11}
\definecolor{airforceblue}{rgb}{0.36, 0.54, 0.66}
\definecolor{lightgray}{rgb}{0.85, 0.85, 0.85}

% set the list style
\lstset{%
	backgroundcolor=\color{lightgray},
	basicstyle=\footnotesize,
	captionpos=b,
	keepspaces=true,
	frame=single,
	framesep=10pt}

% Set the font to sans serif
\renewcommand{\familydefault}{\sfdefault}

% agents that will be used and should be highlighted
\newcommand{\terminal}{\textbf{\color{cornellred} Terminal }}
\newcommand{\device}{\textbf{ \color{airforceblue} Device }}
\newcommand{\user}{\textbf{\color{gray} User }}

\begin{document}

\author{Michael Hawkins}
\title{%
	Static Discharge\\
	\large Bitcoin payments method over\\
	Near Field Communication (NFC)}

\maketitle

\section{Overview}
Near Field Communication (NFC) is used for contactless payment in bank cards and
more recently it has been integrated into mobile phones for use in apps such as
google or apple pay.\\
This project provides a secure transport layer for point of sale lightning
transactions by connecting devices over NFC and interfacing with existing
lightning wallet implementations.

\section{Security}
Lightning point of sale transactions are normally conducted via QR Codes which
provide a one-way interface, encoding a lightning invoice that is then
transferred optically between devices.  The invoice is encoded as plain text
and cannot be encrypted using a PKI due to the inability to exchange keys.  This
means that the invoice can be intercepted and decoded by any QR Code reader with
visibility of the genrated code.\\

NFC provides an interface for bi-directional communication between devices which
enables the use of encryption via key exhange.  This can be added to the
transport of transaction data and greatly improve the security model of making a
lightning payment through the range that the transaction data is visible and the
ability to read the transaction data if it was intercepted.\\

\section{Use Case}
A real world example of a MFC point of sale system, we can look at systems that
require a successful payment to gain access to transport.  Transport For London
use a system of NFC readers to allow travellers to pay for travel via NFC as an
alternative to purchasing traditional paper tickets.\\

Users present their NFC device to a terminal that attempts to make a transaction
that, if successful, unlocks the turnstile and allows them access to the
train platform or informs the bus driver that payment has been received.\\

\section{objectives}
From a design point of view we have a service (access to train platform) and 3
actors:
\begin{itemize}
    \item \textbf{USER} the entity that requests a service
    \item \textbf{DEVICE} used by the \textbf{USER} to interact with the service
    \item \textbf{TERMINAL} that provides the service
\end{itemize}

To break the system down into the simplest form requires the following:
\begin{itemize}
	\item \textbf{TERMINAL} MUST generate invoice
	\item \textbf{DEVICE} MUST read an invoice
	\item \textbf{DEVICE} MUST crate a lightning transaction using the invoice
	\item \textbf{USER} SHOULD be notified of the transaction
	\item \textbf{TERMINAL} SHOULD confirm the transaction
\end{itemize}

Constraints on the system are:
\begin{itemize}
    \item communication between \textbf{TERMINAL} and \textbf{DEVICE} MUST be over NFC
    \item lightning invoices MUST be unique
	\item communications SHOULD be encrypted
	\item transactions SHOULD be made without \textbf{USER} input
\end{itemize}

\section{Technical Specification}

\subsection{\textbf{TERMINAL}}
\subsubsection{NFC}

\begin{quote}
I had initially purchased the RC522 module for the raspberry pi thinking that I
would be able to communicateusing that.  It turns out that RC522 is an old
chip and it doesn't support communication with Android devices.  A last minute
google and I find out that I need a PN532 chip for communication with Android
over NFC!
\end{quote}



\subsubsection{Bitcoin}
The Raspberry Pi runs a \href{https://github.com/bitcoin/bitcoin}{bitcoin core} node
connected to the testnet3 network.

\subsubsection{LND}
An instance of \href(https://github.com/lightningnetwork/lnd){LND} is also running as
a backend for the mobile wallet\footnote{This was a comprimise as I had wanted a
standalone app for the mobile device but the only wallet I could get running that I
could easily develop with was the LN-Zap wallet}

The LND node was configured to avoid conflicting ports with the c-lightning instance,
using the bitcoind backend, connected to the testnet.  It was also configured to
listen on the LAN network address (10.6.5.14) for rpc commands as the zap wallet
will be using it.

In order to link the Zap wallet to the LND node I also had to install
\href{https://github.com/LN-Zap/lndconnect}{lndconnect} in order to generate the
lnconnect url needed (sigh).  Once installed I found that I couldn't generate a
readable form on the command line so needed to run the following command and then
generate a \href{https://www.online-qrcode-generator.com/}{QR Code online}

\begin{lstlisting}[%
	caption=Query to generate lnconnect url using lndconnect in bash shell,
	language=bash]
MACAROON_HOME=~/.lnd/data/chain/bitcoin/testnet
./bin/lndconnect -j \
	--lnddir ~/.lnd/ \
	--host 10.6.5.14 \
	--adminmacaroonpath=\${MACAROON\_HOME}/admin.macaroon \
	--datadir=~/.lnd/data \
	--tlscertpath=~/.lnd/tls.cert
\end{lstlisting}

\subsubsection{C-Lightning}
An instance of \href{https://github.com/ElementsProject/lightning}{c-lightning} is
running and connected to the local bitcoind instance.

To simplify the network, the node is connected to the LND instance as a peer and
bilateral channels open between the two to avoid any routing issues during testing
and demonstrations.

\subsubsection{Invoice Provider}
We use python and the library pylighting to interact with the lightningd instance and
provide the methods needed to generate an invoice.

\subsection{Mobile Hardware}
The mobile device used for development is a oneplus A5000 running Android 9.  This has
developer mode and NFC enabled:
\begin{quote}
Settings > Bluetooth and device connection > Connection preferences > NFC
\end{quote}

\subsection{Mobile Software}
\subsubsection{Wallet Selection}
To avoid having to create a lightning wallet, the functionality for communicating over
NFC is built into the open source \href{https://github.com/LN-Zap/zap-android}{Zap}
wallet.  The advantage that this wallet had over the other wallets I evaluated was
simply that I could compile it!

I was suprised by the relatively small selection of mobile wallets available and
even more so by the problems that I had in being able to set up a development
environment to work on them.

\begin{itemize}
\item Eclair Wallet would have been my first choice as it is a full lightning node
implementation in Java that I'm comfortable programming with.  Unfortunately getting it
to build in android studio was beyond me.  I believe that this was due to eclair-core
being written in Scala which I was able to compile successfully but when including the
jar it looked like I was missing some of the standard library.

\item \href{https://github.com/btcontract/lnwallet}{Bitcoin Lightning Wallet} is written
in Scala so after my experience with eclair I was put off somewhat and having never
writen in Scala before I thought that the learning curve would be too steep to get going
with using it in the hackathon.

\item \href{https://github.com/bluewallet/bluewallet}{Bluewallet} is wallet written in
Javascript (nodejs) that I wasn't able to build. I think that there was/is a problem
with my development environment that stops me from running react-native applications.

\item Lightning labs \href{https://github.com/lightninglabs/lightning-app}{mobile wallet}
was written in Javascript and the build instructions lean heavily towards development
using Apple Mac (I'm rocking linux). Once again not able to run this due to my dev
environment.

\item \href{}{Spark} is another wallet that was on my radar but never tried this as it
wasn't a complete lightning node and is written in Javascript, both of which put me off.
\end{itemize}

\subsubsection{Modifications}
\verb|SendPaymentActivity.java| is the class containing the methods for sending the
lightning (and bitcoin) payments.
\verb|ScanActivity| is the class that is used as a template for creating the intent as
it defines the QR interface.  We'll be creating the NFC interface which is a stripped
down version of this.

\subsection{USER}
The \textbf{USER} should have minimal input in the interaction.  A constraint of NFC on Android
is that the scrren needs to be on for NFC, this the user will do before presentig the
\textbf{DEVICE} to the \textbf{DEVICE}.

The \textbf{DEVICE} SHOULD show a notification to let the \textbf{USER} know that an interactoin has
taken place.  The standard form of this is with a vibration but it would also be nice to
have an onscreen display of the status of the interaction.

\end{document}
